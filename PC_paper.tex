\documentclass[fleqn,usenatbib]{mnras}

% MNRAS is set in Times font. If you don't have this installed (most LaTeX
% installations will be fine) or prefer the old Computer Modern fonts, comment
% out the following line
\usepackage{newtxtext,newtxmath}
%\usepackage{times}
% Depending on your LaTeX fonts installation, you might get better results with one of these:
% \usepackage{mathptmx}
% \usepackage{txfonts}

% Use vector fonts, so it zooms properly in on-screen viewing software
% Don't change these lines unless you know what you are doing
\usepackage[T1]{fontenc}
\usepackage{ae,aecompl}

%%%%% AUTHORS - PLACE YOUR OWN PACKAGES HERE %%%%%
%\usepackage[colorlinks,urlcolor=blue,citecolor=blue,linkcolor=blue]{hyperref}
\usepackage{color}
\usepackage{graphicx}
\usepackage{amsmath}	% Advanced maths commands
\usepackage{amssymb}	% Extra maths symbols
\usepackage{mathrsfs}	% Extra extra math symbols
\usepackage{natbib}
\usepackage[colorlinks,urlcolor=magenta,citecolor=blue,linkcolor=blue]{hyperref}
\usepackage{multirow}
\usepackage{etoolbox}
\usepackage{subfig}
\usepackage{microtype}
\usepackage{listings}

%%%%% AUTHORS - PLACE YOUR OWN COMMANDS HERE %%%%%
%%% Fields %%%
\newcommand{\hdf}{HDF-N}
\newcommand{\hdfn}{HDF-N}
\newcommand{\hdfs}{HDF-S}
\newcommand{\cdfs}{CDF-S}

%%% Telescopes %%%
\newcommand{\hst}{\textit{HST}}
\newcommand{\iras}{\textit{IRAS}}
\newcommand{\iso}{\textit{ISO}}
\newcommand{\spitzer}{\textit{Spitzer}}
\newcommand{\sirtf}{\textit{Spitzer}}
\newcommand{\chandra}{\textit{Chandra}}
\newcommand{\planck}{\textit{Planck}}

%%% Filters %%%
\newcommand{\wfu}{\hbox{$\mathrm{U}_{300}$}}
\newcommand{\wfb}{\hbox{$\mathrm{B}_{450}$}}
\newcommand{\wfv}{\hbox{$\mathrm{V}_{606}$}}
\newcommand{\wfi}{\hbox{$\mathrm{I}_{814}$}}
\newcommand{\acsb}{\hbox{$\mathrm{B}_{435}$}}
\newcommand{\acsv}{\hbox{$\mathrm{V}_{606}$}}
\newcommand{\acsi}{\hbox{$i_{775}$}}
\newcommand{\acsz}{\hbox{$z_{850}$}}
\newcommand{\nicj}{\hbox{$\mathrm{J}_{110}$}}
\newcommand{\nich}{\hbox{$\mathrm{H}_{160}$}}
\newcommand{\wfcy}{\hbox{$\mathrm{Y}_{105}$}}
\newcommand{\wfcj}{\hbox{$\mathrm{J}_{125}$}}
%\newcommand{\wfcj}{\hbox{$J_{110}$}}
\newcommand{\wfch}{\hbox{$\mathrm{H}_{160}$}}
\newcommand{\sdssu}{\hbox{$u$}}
\newcommand{\sdssg}{\hbox{$g$}}
\newcommand{\sdssr}{\hbox{$r$}}
\newcommand{\sdssi}{\hbox{$i$}}
\newcommand{\sdssz}{\hbox{$z$}}
\newcommand{\mone}{\hbox{$[3.6]$}}
\newcommand{\mtwo}{\hbox{$[4.5]$}}
\newcommand{\mthree}{\hbox{$[5.8]$}}
\newcommand{\mfour}{\hbox{$[8.0]$}}
%\newcommand{\mone}{\hbox{$[3.6\mu\mathrm{m}]$}}
%\newcommand{\mtwo}{\hbox{$[4.5\mu\mathrm{m}]$}}
%\newcommand{\mthree}{\hbox{$[5.8\mu\mathrm{m}]$}}
%\newcommand{\mfour}{\hbox{$[8.0\mu\mathrm{m}]$}}

%%% Astronomy Abreviations %%%
\newcommand{\mstar}{\hbox{M$_{\star}$}}
\newcommand{\lstar}{\hbox{L$_{\star}$}}
\newcommand{\Msol}{\hbox{$\mathrm{M}_\odot$}}
\newcommand{\msol}{\hbox{$\mathrm{M}_\odot$}}
\newcommand{\Zsol}{\hbox{$Z_\odot$}}
\newcommand{\zsol}{\hbox{$Z_\odot$}}
\newcommand{\Lsol}{\hbox{$L_\odot$}}
\newcommand{\lsol}{\hbox{$L_\odot$}}
\newcommand{\lir}{\hbox{$L_{\mathrm{IR}}$}}
\newcommand{\zph}{\hbox{$z_\mathrm{ph}$}}
\newcommand{\zphot}{\hbox{$z_\mathrm{ph}$}}
\newcommand{\lbol}{\hbox{$L_\mathrm{bol}$}}
\newcommand{\snr}{\hbox{$\mathrm{S/N}$}}
\newcommand{\reff}{\hbox{$r_\mathrm{eff}$}}
\newcommand{\ks}{\hbox{$K_s$}}
\newcommand{\AAA}{\hbox{\AA}}

%%% Spectrum Lines %%%
\newcommand{\lya}{ Ly$\alpha \;$}
\newcommand{\lyb}{Lyman~$\beta$}
\newcommand{\hb}{\hbox{H$\beta$}}
\newcommand{\ha}{\hbox{H$\alpha$}}
\newcommand{\paa}{\hbox{Pa$\alpha$}}

%%% Units %%%
\newcommand{\kms}{\hbox{km~s$^{-1}$}}
\newcommand{\cms}{\hbox{cm~s$^{-1}$}}
\newcommand{\ergscm}{\hbox{erg~s$^{-1}$~cm$^{-2}$}}
\newcommand{\cnts}{\hbox{cnt~s$^{-1}$}}
\newcommand{\uJy}{\hbox{$\mu$Jy}}
\newcommand{\ujy}{\hbox{$\mu$Jy}}
\newcommand{\degree}{\hbox{$^\circ$}}
\newcommand{\degsq}{\hbox{degree$^2$}}
\newcommand{\arcminsq}{\hbox{arcmin$^2$}}
\newcommand{\um}{\hbox{$\mu$m}}

%%% per Units %%%
\newcommand{\perarcminsq}{\hbox{arcmin$^{-2}$}}
\newcommand{\perdegsq}{\hbox{degree$^{-2}$}}
\newcommand{\permpc}{\hbox{Mpc$^{-1}$}}
\newcommand{\permpcsq}{\hbox{Mpc$^{-2}$}}
\newcommand{\permpccu}{\hbox{Mpc$^{-3}$}}
\newcommand{\percmsq}{\hbox{cm$^{-2}$}}
\newcommand{\percmcu}{\hbox{cm$^{-3}$}}
\newcommand{\perpixel}{\hbox{pixel$^{-1}$}}

%%% Math %%%
\newcommand{\lsim}{\lesssim}
\newcommand{\gsim}{\gtrsim}
\newcommand{\mathS}{\hbox{$\mathcal{S}$}}
\newcommand{\mathR}{\hbox{$\mathcal{R}$}}
\newcommand{\mathM}{\hbox{$\mathcal{M}$}}
\newcommand{\mcal}{\hbox{$\mathcal{M}$}}
\newcommand{\rcal}{\hbox{$\mathcal{R}$}}
\newcommand{\scal}{\hbox{$\mathcal{S}$}}
\newcommand{\infinity}{\hbox{$\infty$}}
\newcommand{\err}[2]{$^{+#2}_{-#1}$}

%%% General %%%
\newcommand{\etal}{et al.}
\newcommand{\eg}{e.g.}
\newcommand{\ie}{i.e.}
\newcommand{\cf}{cf.}
% \newcommand{\ion}[2]{\hbox{#1$\;${\small\rm{#2}}}}
\newcommand{\mybullet}{\noindent$\bullet$}
\newcommand{\uit}{\textit{UIT}}
\newcommand{\nd}{...}
%\newcommand{\cmodel}{\hbox{\tt cmodel}}
%\newcommand{\bs}{\hbox{$\!\!\!\!$}}
\newcommand{\todo}[1]{{\tt #1}}
\newcommand{\citeeg}[1]{(\eg, \citealt{#1})}
\newcommand{\ignore}[1]{}

%%% Extra %%%
% \newcommand{\farcm}{\mbox{\ensuremath{.\mkern-4mu^\prime}}}    % fractional arcminute symbol: 0.'0
% \newcommand{\farcs}{\mbox{\ensuremath{.\!\!^{\prime\prime}}}}  % fractional arcsecond symbol: 0.''0
% \newcommand{\fdg}{\mbox{\ensuremath{.\!\!^\circ}}}             % fractional degree symbol:    0.°0
%\newcommand{\arcdeg}{\ensuremath{^{\circ}}}%                    % degree symbol:  °
% \newcommand{\sun}{\ensuremath{\odot}}%                         % sun symbol
% \newcommand{\apj}{ApJ}%                                        % Journal abbreviations
% \newcommand{\apjs}{ApJS}
% \newcommand{\apjl}{ApJL}
% \newcommand{\aap}{A{\&}A}
% \newcommand{\aaps}{A{\&}AS}
% \newcommand{\mnras}{MNRAS}
% \newcommand{\aj}{AJ}
% \newcommand{\araa}{ARAA}
% \newcommand{\pasp}{PASP}
\newcommand{\Teff}{\ensuremath{T_{\mathrm{eff}}}}%               % T_eff
\newcommand{\logg}{\ensuremath{\log g}}%                         % log g
%\newcommand{\bv}{\ensuremath{B\!-\!V}}%                         % B-V
%\newcommand{\ub}{\ensuremath{U\!-\!B}}%                         % U-B
%\newcommand{\vr}{\ensuremath{V\!-\!R}}%                         % V-R
%\newcommand{\ur}{\ensuremath{U\!-\!R}}%                         % U-R
%\newcommand\ion[2]{#1$\;${\scshape{#2}}}%                       % ion, i.e., CII = \ion{C}{ii}


\newcommand{\editorial}[1]{\textcolor{red}{#1}}
\newcommand{\multic}[2]{\multicolumn{#1}{c}{#2}}
\newcommand{\rottext}[2]{\multirow{#1}{*}{\rotatebox[origin=c]{90}{#2}}}

%%%%%%%%%%%%%%%%%%% TITLE PAGE %%%%%%%%%%%%%%%%%%%
\title[Planck Cluster Paper]{Planck Cluster Paper}

% The list of authors, and the short list which is used in the headers.
% If you need two or more lines of authors, add an extra line using \newauthor
\author[S. Boada et al.]
{\parbox{\textwidth}{Steven~Boada,$^{1}$\thanks{E-mail: boada@physics.rutgers.edu}
JPH,$^{1}$, Felipe
}\vspace{0.4cm}\
\\
\parbox{\textwidth}{$^{1}$Physics and Astronomy Department, Rutgers University, Piscataway, NJ 08854-8019, USA}}

% These dates will be filled out by the publisher
\date{Accepted XXX. Received YYY; in original form ZZZ}

% Enter the current year, for the copyright statements etc.
\pubyear{2017}

% Don't change these lines
\begin{document}
\label{firstpage}
\pagerange{
\pageref{firstpage}--
\pageref{lastpage}}
\maketitle

\begin{abstract}
	\noindent We propose to continue our program of optical imaging to unveil all of the most massive clusters in
	the observable Universe. We start from the all-sky Planck Sunyaev-Zel’dovich (SZ) catalogs, which
	contain several hundred high significance (signal-to-noise ratio, SNR $> 5$) unconfirmed cluster
	candidates. Since SZ selection favors high mass clusters and the Planck confirmation process favored
	low redshift systems, the highest significance unconfirmed candidates are, therefore, likely massive
	clusters ($M_{500} > 5 ×\times 10^{14}$ \Msol) at relatively high redshift ($z > 0.5$). Our proposed observations,
	using MOSAIC-3 on Mayall, are designed to confirm the presence of a brightest cluster galaxy (to
	$z \sim 1$) and red sequence of accompanying cluster members (to $z \sim 0.7$). Preliminary results from
	our observations over the past two years have validated our approach by the detection of optical
	clusters in a number of Planck candidates, including the discovery of rich systems at $z = 0.553$ and
	$z = 0.830$ that rival the most massive clusters known. The proposed observations represent the first
	step required to provide a complete all-sky census throughout the observable Universe of the most
	massive, high redshift clusters. Their expected high redshift and high mass make the unconfirmed
	Planck clusters, arguably, the most important available sample for probing deviations from $\Lambda$CDM
	and defining the high-mass end of the cluster mass function.
\end{abstract}

\section{Introduction}
Throughout this paper, we adopt the following cosmological model from the Buzzard simulations: $\Omega_\Lambda = 0.714$, $\Omega_M = 0.286$, and $H_0= 70$ \kms \mpc, assume a Chabrier initial mass function (IMF; \citealt{Chabrier2003}), and use AB magnitudes \citep{Oke1974}.

\section{Design}\label{sec:design}
lorem
\subsection{Observations}\label{sec: observations}

The proposed observing strategy consists of targeted $riz$ observations of individual candidates with
exposure times of 350 s, 1100 s and 1100 s (assuming dark conditions) to provide $5\sigma$ detections
limits of $r = 24.5$, $i = 24.5$, $z = 24.2$ ensuring the unambiguous detection of the faint galaxies
(i.e., 0.4L$_∗\star$ ) in the red cluster sequence up to $z \sim 1.0$ (see Fig. 3) and of BCGs to higher redshifts.
The choice of filters in our program is driven by the need to segregate early-type galaxies in the
cluster through their colors (or photometric redshifts) by sampling blue-ward and red-ward of the
4000\AA\ break

For these observations we will follow the successful setup we used before. We obtained an hour
on each target in the Ks band using 1 minute exposures (5 coadded 12 s exposures) taken at
60 different dither positions distributed quasi-randomly over a square 100'' × 100'' region. This
produced reduced images with uniform exposure and sky level. The large FOV of the dithered NIR
images (approximately 28' × 28' ) comfortably matches the MOSAIC observations proposed for Run
1.
Based on our experience over the last two years, an hour of integration on NEWFIRM will allow us
to reach a magnitude limit of Ks $\sim 22.0$ (AB, $3\sigma$). This magnitude limit corresponds to $\sim M_{\star} + 2$
in the cluster luminosity function at $z = 1.0$ as measured by De Propris et al. (1999), and assuming
K AB = K Vega $+ 1.86$. This surface brightness limit corresponds to $\sim M_{\star} + 1.0$ at $z = 1.5$, sufficient
for detecting sub L$_{\star}$ at this limit. This allows for confident detection of the BCG and associated red
cluster sequence.

The narrow-band observations were made with the Mosaic cameras mounted at the
prime foci of the KPNO and CTIO 4-m telescopes. The instruments and basic reduction
procedures are given in detail in Paper I. Here we note that each camera consists of a 2x4
array of 2048x4096 SITe CCDs. The field of view is 36’x36’, and the plate-scale of the
final reduced images is 0.27” pixel−1
The target galaxies were imaged using the Mosaic camera on
the KPNO Mayall 4 m telescope. The Mosaic camera consists of
eight 2048×4096 CCDs separated by a small gap (∼50 pixels).
On the Mayall telescope, the imager has a 36' × 36' field of
view with 0.26'' pixels.

the near-IR with the National Optical Astronomy
Observatory (NOAO) Extremely Wide-Field Infrared Imager
(NEWFIRM)

The
NEWFIRM camera, designed to quickly map large areas of the
sky, contains four InSb 2048 × 2048 pixel arrays arranged in a
2 × 2 pattern with a 28' field of view and an approximately
1' gap between the CCDs. The detector has a pixel scale of
0.4'' pixel−1. \citep{Probst2004}

\section{Data Reduction and Calibration}\label{sec:data reduction}
Standard processing of dark frame subtraction, flat fielding,
sky-subtraction, and bad pixel masking was performed by
the NEWFIRM Science Pipeline (Dickinson \& Valdes 2009;
Swaters et al. 2009) to produce five stacked composite images
for each near-IR band. The final stacked images retain the
detector pixel scale of 0.4'' pixel−1. Variable seeing conditions
caused the final observed point source FWHM in the stacked
images to range from approximately 0.9–1.5''. The H-band
observations were taken during the best seeing conditions and
have the smallest FWHM, and the Ks images have the largest
FWHM. Even in the most crowded portions of the region
observed, however, the difference in effective resolution at the
different NEWFIRM bands has a minimal effect on later source matching.

\subsection{Mosaicking}
Mosaics are created with \textsc{swarp} \citep{Bertin2002}.

\subsection{Astrometric Calibration}
Images are calibrated with \textsc{scamp} \citep{Bertin2006} and \textsc{photometrypipeline} (PP; \citealt{Mommert2017}).

\subsection{Photometric Calibration}
\textit{Sloan Digital Sky Survey} (SDSS; \citealt{York2000}).
We use the SDSS Data Release 12 \citep{Alam2015}
We use The Panoramic Survey Telescope and Rapid Response System (Pan-STARRS) Data Release 1 \citep{Chambers2016, Flewelling2016}.

\section{Analysis}\label{sec:analysis}
lorem

\subsection{Source Extraction}
We used \textsc{SExtractor} \citep{Bertin1996}.

\subsection{Photometric Redshifts}
We determine photometric redshifts from the four-band optical images using BPZ \citep{Benitez2000} following the same
procedure as in Menanteau et al. (2008).

\subsection{Cluster Finding}
We create RGB images using \textsc{stiff} \citep{Bertin2011}.
We use \textsc{MaxBCG} \citep{Koester2007b}.

\section{Results and Discussion}\label{sec:results}

lorem

\section{Summary}\label{sec:summary}

lorem

\section*{Acknowledgements} This research made use of \textsc{APLpy}, an open-source plotting package for Python hosted at \url{http://aplpy.github.com}; the \textsc{IPython} package \citep{Perez2007}; \textsc{matplotlib}, a Python library for publication quality graphics \citep{Hunter2007}. \textsc{iraf} is distributed by the National Optical Astronomy Observatory, which is operated by the Association of Universities for Research in Astronomy under cooperative agreement with the National Science Foundation \citep{Tody1993}. \textsc{PyRAF} is a product of the Space Telescope Science Institute, which is operated by AURA for NASA. Funding for the SDSS and SDSS-II has been provided by the Alfred P. Sloan Foundation, the Participating Institutions, the National Science Foundation, the U.S. Department of Energy, the National Aeronautics and Space Administration, the Japanese Monbukagakusho, the Max Planck Society, and the Higher Education Funding Council for England. The SDSS Web Site is \url{http://www.sdss.org/}. The SDSS is managed by the Astrophysical Research Consortium for the Participating Institutions.

%%%%%%%%%%%%%%%%%%%% REFERENCES %%%%%%%%%%%%%%%%%%
% The best way to enter references is to use BibTeX:
\bibliographystyle{apj}
\bibliography{master}

% if your bibtex file is called example.bib
%%%%%%%%%%%%%%%%% APPENDICES %%%%%%%%%%%%%%%%%%%%%


% Don't change these lines
\bsp

% typesetting comment
\label{lastpage}
\end{document}
